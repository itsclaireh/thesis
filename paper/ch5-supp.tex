\section{Mitigating Sexual Harassment}

The context for this research topic is to explore possible avenues in which sexual harassment could be mitigated or reduced. This context was to be explored by finding new trends among the tweets themselves or by implementing the classifier online to assist in mitigating sexual harassment through technology. Research into sexual harassment and harassment online guided the development of a survey questionnaire that was incorporated into the direction of this project.

\subsection{Literature Review}

Existing papers and their limitations.

\subsection{Other Problems}

Other issues exist that have not been studied in depth, and therefore a quantifiable impact or degree of severity of the issue has not been quantifiable measured. Regardless, some of these issues are worth nothing.

Woman, Action, and the Media (WAM!) created a team in 2015 to assist Twitter in evaluating harassment that took place within the website. While this tool does not look to analyze whether or not a tweet may be of a harassing nature, this report afforded insight into Twitter's system that could potentially have use in other contexts. Serious security flaws exist in the lack of maintenance invested on archiving and storing all tweets and messages posted by users, particularly those of a harassing nature, and this results in two majors problems. First, it eliminates the evidence that victims have to defend themselves; secondly, Twitter's own reporting system \textit{only} accepts live versions of the tweet, further reducing victims' options.

Twitter cannot remove a tweet of a harassing nature if the evidence submitted is a screenshot--the tweet must still be available to be viewed live to be used as evidence when determining whether or not to take action against a Twitter user. More importantly, this report also reveals that when a harasser deletes his or her own tweet, or Twitter removes the offending tweet, that tweet is no longer available to law enforcement agencies ~\cite{matias2015reporting}.

Historically, harassing or stalking types of behavior occurred through phone calls, cards, or other more direct forms of inserting a presence in the victim's life. Now, it is quite common for harassment (both sexual harassment and not) to persist over social media. WAM!'s report indicated that the lack of persisting data and messages over social media platforms is a disservice to victims as it often leaves them without evidence to take to the police and without evidence to get their harasser removed.

Twitter is not the only social media platform that has this problem. Instagram users have the option of deleting their comments and direct messages as well as permanently hiding their presence from a user entirely, including their username. Instagram's public policy regarding their cooperation with law enforcement states that they will comply with valid subpoenas, court orders, or warrants under outlined circumstances. However, if a victim were to be harassed over Instagram and that user removed their messages and presence from the perspective of the victim, the victim would have no evidence to bring to law enforcement in order to file charges and proceed with a legal request for the data. The lack of persistence and lack of archiving disenfranchises victims of sexual and non-sexual harassment on social media.

\section{Survey}

In order to further develop an understanding of how to mitigate sexual harassment, a questionnaire was developed. The questionnaire was reviewed by IRB and found not to need approval, as it dealt with voluntary participants above the age of 18, no identifying information was collected, and responses were completely anonymous.

\subsection{Questionnaire}

Answer choices marked with an asterisk (*) indicates that the answer choice had a text box where the participant could type an answer.
\hfill \break

\begin{adjustwidth}{0cm}{}
    \begin{itemize}
        \item [1)] What is your age?*
        \item [2)] What is your gender identity?
        \begin{itemize}
            \item Male
            \item Female
            \item Other*
            \item Prefer not to respond
        \end{itemize}
        \item [3)] Select all of the following behaviors that you consider to be forms of sexual harassment and the degree of severity you consider the infringement to be.
    \end{itemize}

    \end{adjustwidth}
    \begin{adjustwidth*}{-10em}{}

\begin{table}[H]
    \centering
    \rowcolors{2}{Gray!20}{White}
    \begin{tabular}{m{5.5cm} m{1.5cm} m{1.5cm} m{1.5cm} m{2cm}}
        \toprule
        \rowcolor{White}\textbf{} & {\textbf{}} & {\textbf{}} & {\textbf{}} & \multicolumn{1}{c}{\textbf{Not Sexual}} \\
        \rowcolor{White}\textbf{Action} & \multicolumn{1}{c}{\textbf{Mild}} & \multicolumn{1}{c}{\textbf{Moderate}} & \multicolumn{1}{c}{\textbf{Severe}} & \multicolumn{1}{c}{\textbf{Harassment}} \\
        \midrule
        Staring or leering & \multicolumn{1}{c}{{\Large $\circ$}} & \multicolumn{1}{c}{{\Large $\circ$}} & \multicolumn{1}{c}{{\Large $\circ$}} & \multicolumn{1}{c}{{\Large $\circ$}}\\
        Whistling, catcalling, or winking at someone & \multicolumn{1}{c}{{\Large $\circ$}} & \multicolumn{1}{c}{{\Large $\circ$}} & \multicolumn{1}{c}{{\Large $\circ$}} & \multicolumn{1}{c}{{\Large $\circ$}}\\
        Pinching or poking & \multicolumn{1}{c}{{\Large $\circ$}} & \multicolumn{1}{c}{{\Large $\circ$}} & \multicolumn{1}{c}{{\Large $\circ$}} & \multicolumn{1}{c}{{\Large $\circ$}}\\
        Sexist comments & \multicolumn{1}{c}{{\Large $\circ$}} & \multicolumn{1}{c}{{\Large $\circ$}} & \multicolumn{1}{c}{{\Large $\circ$}} & \multicolumn{1}{c}{{\Large $\circ$}}\\
        Inappropriate drawings & \multicolumn{1}{c}{{\Large $\circ$}} & \multicolumn{1}{c}{{\Large $\circ$}} & \multicolumn{1}{c}{{\Large $\circ$}} & \multicolumn{1}{c}{{\Large $\circ$}}\\
        Messages on social media from strangers & \multicolumn{1}{c}{{\Large $\circ$}} & \multicolumn{1}{c}{{\Large $\circ$}} & \multicolumn{1}{c}{{\Large $\circ$}} & \multicolumn{1}{c}{{\Large $\circ$}}\\
        Making lewd/sexual remarks about someone's looks or body & \multicolumn{1}{c}{{\Large $\circ$}} & \multicolumn{1}{c}{{\Large $\circ$}} & \multicolumn{1}{c}{{\Large $\circ$}} & \multicolumn{1}{c}{{\Large $\circ$}}\\
        Obscene gestures or sounds & \multicolumn{1}{c}{{\Large $\circ$}} & \multicolumn{1}{c}{{\Large $\circ$}} & \multicolumn{1}{c}{{\Large $\circ$}} & \multicolumn{1}{c}{{\Large $\circ$}}\\
        Sending repeated messages, calls, or other forms of contact after the receiver expresses disinterest & \multicolumn{1}{c}{{\Large $\circ$}} & \multicolumn{1}{c}{{\Large $\circ$}} & \multicolumn{1}{c}{{\Large $\circ$}} & \multicolumn{1}{c}{{\Large $\circ$}}\\
        Asking overly personal questions & \multicolumn{1}{c}{{\Large $\circ$}} & \multicolumn{1}{c}{{\Large $\circ$}} & \multicolumn{1}{c}{{\Large $\circ$}} & \multicolumn{1}{c}{{\Large $\circ$}}\\
        Stalking or Harassment & \multicolumn{1}{c}{{\Large $\circ$}} & \multicolumn{1}{c}{{\Large $\circ$}} & \multicolumn{1}{c}{{\Large $\circ$}} & \multicolumn{1}{c}{{\Large $\circ$}}\\
        Groping or other touching & \multicolumn{1}{c}{{\Large $\circ$}} & \multicolumn{1}{c}{{\Large $\circ$}} & \multicolumn{1}{c}{{\Large $\circ$}} & \multicolumn{1}{c}{{\Large $\circ$}}\\
        Sending unsolicited photos of private body areas & \multicolumn{1}{c}{{\Large $\circ$}} & \multicolumn{1}{c}{{\Large $\circ$}} & \multicolumn{1}{c}{{\Large $\circ$}} & \multicolumn{1}{c}{{\Large $\circ$}}\\
        Other (please specify)* & \multicolumn{1}{c}{{\Large $\circ$}} & \multicolumn{1}{c}{{\Large $\circ$}} & \multicolumn{1}{c}{{\Large $\circ$}} & \multicolumn{1}{c}{{\Large $\circ$}}\\
        \bottomrule
    \end{tabular}
\end{table}

\end{adjustwidth*}

    \begin{adjustwidth}{0cm}{}

    \begin{itemize}
        \vspace{-10pt} \item [4)] What social network platform do you use the most frequently?
         \begin{itemize}
            \item Facebook
            \item Youtube
            \item Instagram
            \item Twitter
            \item Reddit
            \item Tumblr
            \item Other (please specify)*
        \end{itemize}
        \item [5)] Considering the social media network designated above, how would you rate this social network platform in terms of its ability \textbf{to prevent} sexual harassment from happening?
        \begin{itemize}
            \item It does not have any mechanism to prevent sexual harassment from happening
            \item It has some mechanisms but these are not adequate
            \item It has a good set of mechanisms to prevent sexual harassment from happening
            \item I don't know
        \end{itemize}
        \item [6)] Considering the social media network designated above, how would you rate this social network platform in terms of convenience of \textbf{reporting sexual harassment}?
         \begin{itemize}
            \item It does not have any mechanism to report an instance of sexual harassment
            \item It has some mechanisms but these are not adequate
            \item It has a good set of mechanisms to report sexual harassment
            \item I don't know
        \end{itemize}
        \item [7)] In your opinion, in what ways could this social network platform be designed to prevent sexual harassment from happening? Mark all that apply.
         \begin{itemize}
            \item Increasing the prevalence/priority advertisement which addresses sexual harassment
            \item Stricter regulation of posts/content on social media sites
            \item Adjustments to news feed algorithms to prioritize content regarding victims of sexual harassment
            \item Automatically archiving or documenting instances of sexual harassment online for evidence
            \item Connecting victims on social media with help (local authorities, other victims, or local resource options)
            \item Please describe any other thoughts you have on ways technology could mitigate sexual harassment:*
        \end{itemize}

    \end{itemize}
\end{adjustwidth}

\section{Findings}

No results at this time.
