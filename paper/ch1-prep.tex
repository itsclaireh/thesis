This chapter addresses the research and work conducted before the development process. It describes sexual harassment as it has developed and recognized both legally and psychologically as well as a discussion of the literature regarding sexual harassment categorization. This was fundamental to designing and applying a strong, comprehensive categorization topology to the dataset. This chapter also discusses the data used in the program, how it was collected, and how it was manually categorized for the supervised machine learning process.

\section{Legal History of Sexual Harassment in the United States}

Defining different types of sexual harassment is a task that remains incomplete despite steady, incremental progress because ``sexual harassment" didn't exist as a criminal offense until it was first supported through case law. Initially, the Title VII of the Civil Rights Act of 1964 made it illegal to discriminate employment opportunities based upon gender (and Title IX later upholding the same philosophies within education). Throughout the 1970's, several women used Title VII provisions to sue their employers for coercing (or attempting to coerce) them into sexual acts, and their successes within the courts enshrined sexual harassment as a criminal offense under these provisions. Since these cases, appropriately defining sexual harassment continues to be refined through improvements in research, laws, and court decisions today. As a controversial topic with many nuances and varied perspectives, proper categorization of sexual harassment necessitates a comprehensive legal understanding alongside a psychological one.

The first Supreme Court case ruling in favor of the victim alleging sexual harassment was in the 1974 case \textit{Barnes v. Train}, in which the harasser was found at fault for firing a female employee for refusing his sexual advances although the term "sexual harassment" was not used yet at this time. A few years later, the Supreme Court upheld that this type of behavior from employers was a violation of Title VII, and subsequently the Equal Employment Opportunity Commission (EEOC) refined the rules to explicitly cover this type of sexual harassment. The first case to grant relief for sexual harassment explicitly under Title VII provisions was \textit{Williams v. Saxbe} (1976). Because of the legal distinction and precedents set, the plaintiff in the landmark case \textit{Meritor Savings Bank v. Vinson} (1986) effectively established \textit{quid pro quo} behaviors as a form of sexual harassment where the plaintiff won a case against her employer who attempted unsuccessfully to coerce her into sexual acts. This case also recognized that verbal remarks and questions of a \textit{quid pro quo} nature were in violation of Title VII even if the victim suffered no tangible consequences because the language and attempt itself created a "hostile work environment" ~\cite{bostonLawReview}. Now, legally recognized instances of sexual harassment continue to fall under the domain of the the Equal Employment Opportunity Commission (EEOC) and Title VII, and the accepted definitions categorize sexual harassment as either being \textit{quid pro quo} acts or behaviors that contribute to a ``hostile work environment" ~\cite{legalHistory1988}. Through a myriad of rule changes, public statements, and Supreme Court decisions, the Department of Education's Office for Civil Rights has uphold these same principles as they pertain to education because sex-motivated violence or harassment to a student creates a hostile environment.

What behaviors qualify as sexual harassment as a criminal offense are continuously revisited through new court decisions. Under U.S. law, \textit{quid pro quo} sexual harassment, also referred to as sexual bribery, includes attempted and actual pursuits of a sexual nature against a person in a professional or academic environment when tangible benefits could be given or denied to the victim. For a long time, researchers overestimated the frequency of this type of harassment compared to other forms  because sexually explicit cases have historically been preferred over sex-related work ~\cite{fitzgerald1995measuring}.

Actions that contribute to a "hostile work environment" are the much more common category of sexual harassment. Less explicitly sexual behaviors, such as offhand remarks, teasing or banter, and personal questions are not explicitly banned as acts of sexual harassment. However, if these acts occur with a sufficient degree of severity or frequency, they would then then be considered as elements that create a hostile environment in work or school and therefore qualify as sexual harassment. When evaluating behaviors that do not take place in an educational or professional setting, the actions cannot be evaluated under Title VII or Title IX regulations. These instances of sexual harassment that don't occur in a professional setting must occur with such a degree of frequency or severity that they can be considered (with context) as falling under harassment, stalking, cyberstalking, sexual assault, or other criminal laws. This technicality makes proper evaluation of sexual harassment complex because the behavior in question, while it might be considered sexual harassment in a professional setting, does not have an equal opportunity regulation to dictate it as such when the action occurs between peers.

\section{Literature Review}

As the courts first began to address more and more claims of alleged sexual harassment, the need for a proper categorization grew in order to appropriately assess the degree and severity of the actions. From both a legal and social research perspective, the lack of consistency among categorization definitions caused problems when trying to compare and use instances or research from one context as a guide when evaluating another. The first widely used standardization was designed by Frank J. Till in 1980. In his work, Till defines five major categories that consist of generalized sexist remarks or behavior, inappropriate and offensive but essentially sanction-free sexual advances, solicitation, coercion, and sexual crimes ~\cite{till}. This categorization was the most frequently used throughout the 1980's, although most organizations generally either rephrased the categories or consolidated them into three categories.

In 1992, James E. Gruber, a legal consultant, made a significant contribution to categorizing sexual harassment by defining three overarching categories and 3-4 subtypes for each one, for a total of 11 distinct, exhaustive, and mutually exclusive categories. Gruber's first category ``verbal requests" includes sexual bribery, sexual advances, relational advances, and subtle pressures/advances. These are all behaviors that are said directly to the victim with the intent of a sexual or an increased personal or social relationship goal even if the relationship being sought isn't obviously sexual. Subtle pressures and advances generally refers to statements or questions that can be ambiguous and not directed towards the victim but are still inappropriate, such as accidentally ``thinking out loud" or double entendres.

The second category, ``verbal comments", includes personal remarks, subjective objectification, and sexual categorical remarks. These categories generally refer to statements of a nonsolicitory nature directed either to a woman (ex. teasing or jokes), about a woman, or about women in general. Sexist verbal remarks, rumors, inappropriate comments regarding someone's body or perceived level of attractiveness, and bystander harassment are also classified here.

The third category, ``nonverbal displays", includes sexual assault, sexual touching, sexual posturing, and possession/display of inappropriate sexual materials. Actual or attempted coercion is sexual assault regardless. Pinching, grabbing, groping are examples of sexualized touching. Sexual posturing includes violations of personal space and attempts to make physical contact, including behaviors like making an obscene sexual gesture with one's hands. The possession or display of sexual materials includes personal items regarding sexuality (tampons, pads, birth control, sex toys, etc.) in addition to pornographic materials or sexist materials.

Altogether, there are 11 distinct types of sexual harassment, and these categories are both mutually exclusive and reflective of the EEOC's guidelines ~\cite{gruber1992topology}. Gruber originally wrote these categories based upon reviews of sexual harassment that only included female victims and male harassers, but their continued application today shows that these definitions are not gender exclusive.

These mutually exclusive categories establish a fluid progression in how certain behaviors can contribute to a hostile work environment, and Gruber's work continues to influence many more recent approaches to this topic. In a 2005 review of past, present, and future directions of improving gender and minority diversity in professional environments, Murrell and James refer to Gruber's work as a cornerstone in developing a comprehensive legal definition of sexual harassment ~\cite{murrell2002pastpresentfuture}. A study performed in 2005 to evaluate the effect of an obscene television show on individuals' perception of what constitutes sexual harassment used Gruber's categorization in their participant surveys in order to do so ~\cite{berlin2005gruberverification}. Despite the significance of his contribution, there is a neglected space of defining sexual harassment beyond the scope of Title VII and Title IX that Gruber's categories do not accommodate entirely. Consequently, many researchers continue to consolidate these categories. and in doing so, claim a lesser degree of specificity to permit the nuances of peer to peer harassment to be appropriately placed.

A common issue with classifying sexual harassment lies in the varied perceptions of what is and isn't sexual harassment among observers ~\cite{studzinska2015perception}. The victim's perception is integral not just for legal categorization but also to understand the degree and severity of the harm inflicted on those who are exposed to the behavior. Some concerns with relying too heavily on the victim's perception include fear that the victim might be biased and in some rare cases, deliberately dishonest. However, psychological research into coping with sexual harassment suggest that this is not necessarily the case. A study by Aparna Pathak on sexual harassment and coping behaviors synthesized many different research publications over the past few decades in her work. This synthesis notes that a 1997 study found that experiencing sexual harassment, whether or not the victim is aware of it, will still have negative outcomes on the victim ~\cite{schneider1997effects} in terms of health and distress. Additional research into false claims shows that the prevalence of false allegations of sexual assault comprise between 2\% and 10\% of cases, which is the same statistical prevalence as false reporting of non-sexual crimes ~\cite{falseallegations2010}. These types of research supports the research philosophy of giving the victim the benefit for the doubt in some cases.

Additionally, a lack of research exists in sexual harassment as it pertains to instances that don't occur within title VII or Title IX, particularly on the streets. Pathak's review notes that (as of 2015, the time this was written) a study performed in 2000 is one of the only known, peer-reviewed attempt at documenting the extent of unwanted sexual attention from strangers. MacMillan et al. found that over 80\% of women experienced unwanted sexual attention (ex. catcalls) and just under 30\% of women experienced direct confrontation of a sexual nature from strangers ~\cite{macmillan2000street}. Overall, these caveats indicate that while Gruber's categorization might be the most appropriate tool for evaluating sexual harassment in a professional capacity because of it's adherence to EEOC guidelines, it does not necessarily scale towards including social environments outside of a workplace or school.

Shortly after Gruber published his categorization, Fitzgerald et al. developed a simpler, consolidated categorization architecture in order to consequently develop a better questionnaire for measuring sexual harassment. Their categorization was comprised of three types: unwanted sexual attention and gender harassment (hostile work environment) and sexual coercion (\textit{quid pro quo}). This categorization was an attempt to distinguish between sexual harassment "as a legal concept and a psychological construct" in order to better accommodate how a victim might perceive or label a behavior. Accommodating this "gray space" in victim perception allows for more consistency among responses, which guided their research goal of developing a more scalable questionnaire for surveying the frequency of sexual harassment ~\cite{fitzgerald1995measuring}. In a 2008 publication, Chamberlain et al. deviated from Gruber's 11 types for similar reasons. Upon the basis that these legal-driven approaches "underscore diversity" and "suggest substantial variation with regard to intent and severity" ~\cite{chamberlain2008newcategories} from the victims' perception. For their purposes, Chamberlain et al. uses the following sexual harassment categories: patronizing (sexist but nonsexual remarks, gestures, or condescension), taunting (sexual gestures, physical displays, and overly personal comments and queries), and predatory (encompassing sexual solicitation, sexual promises or threats, touching, and forced contact).

\section{Data}

A collection of 10,000 tweets were originally assembled into a single excel sheet with the tweet's unique ID and contents in two columns. The next three column headers contained the checks of relevancy, stance, and type of sexual harassment. If the correct classification could not be determined from the text available, the index was left blank.

\begin{table}[H]
    \centering
    \caption{Categorization Header}
    \rowcolors{2}{Gray!20}{White}
    \begin{tabular}{b{1cm} b{2.5cm} b{2cm} b{2cm} b{4.7cm}}
        \toprule
        \multicolumn{2}{c}{} &
        \multicolumn{3}{c}{Classification Labels}\\
        \cmidrule(lr){3-5}
        \rowcolor{White}& {} & {\textbf{Relevant}} & {\textbf{Stance}} & {\textbf{Harassment Category}} \\
        & {} & {1. Related} & {1. Support} & {1. Patronizing} \\
        \rowcolor{White}& {} & {2. Unrelated} & {2. Against} & {2. Unwanted Sexual Attention} \\
        & {} & {} & {3. Neutral} & {3. Predatory} \\
        \rowcolor{White}\textbf{id\_str} & {\textbf{text}} & {} & {} & {4. Not Enough Context} \\

        \midrule
        1 & `\ldots example \ldots' & \multicolumn{1}{c}{1} & \multicolumn{1}{c}{1} & \multicolumn{1}{c}{3}\\
        2 & `\ldots example \ldots' & \multicolumn{1}{c}{2} & \multicolumn{1}{c}{ } & \multicolumn{1}{c}{ }\\
        3 & `\ldots example \ldots' & \multicolumn{1}{c}{1} & \multicolumn{1}{c}{2} & \multicolumn{1}{c}{ }\\
        \bottomrule
    \end{tabular}
\end{table}


To create a reliable training set of data, a sufficiently large amount of tweets were manually categorized by human readers. These voluntary participants were given a comprehensive explanation of sexual harassment both legally and psychologically, examples of properly categorized tweets, category definitions, as well as the categorization rules.

Participants were each provided 500 unlabeled tweets and asked to categorize them according to the rules provided. One male and one female of a similar age and level of education shared each set of 500 tweets and categorized them without having access to any other responses beyond the examples provided. Because research suggests that males and females assess sexual harassment differently, this process was performed to improve the integrity and consistency of the training set.

5,000 tweets were categorized total, with each set of 500 being categorized by a male and female. Cumulatively, 10,000 categorizations were made. Only the tweets that were categorized identically by both the male and female participant were considered in the training and testing sets of data. 
